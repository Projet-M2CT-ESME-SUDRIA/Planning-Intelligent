\documentclass[12]{article}

\usepackage[T1]{fontenc}
\usepackage[utf8]{inputenc}
\usepackage[french]{babel}
\usepackage[official]{eurosym}
\usepackage{txfonts}
\usepackage{graphicx}
\usepackage{lastpage}
\usepackage{fancyhdr}
\usepackage{titlesec}
\usepackage{color}
\usepackage{tikz}

\begin{document}

	% PRE-PROCESSING %

	\section*{1ère étape : Pré-processing}

	\begin{tikzpicture}

		% CREATION DES DIFFERENTS CARRES %

		\tikzstyle{quadri}=[rectangle,draw,fill=white!50,text=black]

		% COURS %

		\node[quadri] (C) at (-5,3) {
			\begin{minipage}{4cm} 
				\begin{center}
					Classes
				\end{center}
	  			\begin{itemize} 
	  				\item[$\bullet$] Nom (M1A, M1B...) 
					\item[$\bullet$] Series(M1AS1, M1AS2,...)
					\item[$\bullet$] Matières (Liste de cours à faire)
				\end{itemize} 
			\end{minipage} 
		};

		% MATIERES %		

		\node[quadri] (M) at (0,3) {
			\begin{minipage}{4cm} 
				\begin{center}
					Matières
				\end{center}
	  			\begin{itemize} 
	  				\item[$\bullet$] Nom 
					\item[$\bullet$] Nombre d'heure totale
					\item[$\bullet$] Nombre d'heure effectués
					\item[$\bullet$] Liste des salles
				\end{itemize} 
			\end{minipage} 		
		};

		% PROFS %

		\node[quadri] (P) at (5,3) {
			\begin{minipage}{4cm} 
				\begin{center}
					Profs
				\end{center}
	  			\begin{itemize} 
	  				\item[$\bullet$] Nom 
					\item[$\bullet$] Matières enseignées
					\item[$\bullet$] Jours disponnibles
					\item[$\bullet$] Nombre d'heure/semaine
					\item[$\bullet$] Status (permanant ou vacataire)
				\end{itemize} 
			\end{minipage} 		
		};

		% Salles %		

		\node[quadri] (S) at (0,0) {
			\begin{minipage}{4cm} 
				\begin{center}
					Salles
				\end{center}
	  			\begin{itemize} 
	  				\item[$\bullet$] Nom 
				\end{itemize} 
			\end{minipage} 		
		};
		
		% CREATION DES LIENS ENTRE LES CARRES %		

		\tikzstyle{estun}=[->,dotted,very thick,>=latex]
		\draw[estun] (C)--(M);
		\draw[estun] (P)--(M);
		\draw[estun] (S)--(M);

	\end{tikzpicture}

		\paragraph{}
		Dans un premier temps nous allons compléter le table avec l'ensemble des matières. Nous allons donner le nom, l'ensemble des heures qui vont devoir être données et la liste des salles qui vont être adéquates pour donner ce cours. Nous allons différencier une même matière entre un cours magistral et un Tp ou Td. Par exemple nous allons retrover comme matières : Cours Java, Tp Java... Un cours n'a pas besoin de matériel comme des ordinateurs par exemple. Du coup nous pouvons utiliser n'importe quelle salle pouvant acceuillir l'ensemble des élèves de la classe. Nous allons limiter l'effectif d'une classe pour qu'un plus grand nombre de salle puisse les acceuillir.

		\paragraph{}
		Nous allons pouvoir ensuite remplir la table des classes. Les classes doivent avoir un nombre d'élève limité en fonction des salles. Cette limite doit être faite par l'établissement. Dans cette classe nous allons lister l'ensemble des matières qu'elles doivent suivre.\\
Une classe peut avoir des séries pour les cours de Tp par exemple. C'est pourquoi nous allons devoir aussi les lister et indiquer les matières qui vont être faite.\\
!!!!! IL FAUT SAVOIR SI DANS LES CLASSES ONT CREE LES CLASSE DE COURS MAGISTRAUX ET DE TP OU SI ON CREE UNE AUTRE TABLE POUR METTRE LES SERIES QUI HERITERAIS DES CLASSES  !!!!

		\paragraph{}
		Enfin nous allons remplir la table des profs en indiquant leur nom, la ou les matières qu'ils vont enseigner, leurs jours et heures de disponnibilité et enfin savoir si il s'agit de professeurs permanent ou des vacataire venant d'entrepise. Les professeur vacataire seront prioritaire sur les disponibilité par rapport au enseignant permanant lors de la répartition des cours car ils sont censé être moins disponible.

		\paragraph{}
		Dans la première de planification, nous allons dans un premier organiser nos données.	 
Tout d'abord nous allons regarder la liste des matière que doit effectuer une classe. L'objectif va être de donner un prof pour une matière et ceci pour une classe. Nous allons répéter cette action jusqu'à donner pour toutes les matières d'une classe, un prof.
		
		\paragraph{}
		Comme une matière peut être enseigné par plusieurs professeurs, nous allons regarder lors de l'attribution des profs au classe si il donne déja ce même cours à une autre classe ou non. Si c'est le cas, nous allons vois un autre professeur si il y en a un. De même nous observons si oui ou non il donne déja ce cours. Nous répétons cette action jusqu'à trouver un enseignant qui ne donne pas déja ce cours. Si tous les professeurs donne déjà ce cours, on en choisit un au hasard.


	% PLANIFICATION %

	\section*{2ème étape : Planification}
		
		\begin{tikzpicture}

		% CREATION DES DIFFERENTS CARRES %

		\tikzstyle{quadri}=[rectangle,draw,fill=white!50,text=black]

		% COURS %

		\node[quadri] (C) at (-5,3) {
			\begin{minipage}{4cm} 
				\begin{center}
					Cours
				\end{center}
	  			\begin{itemize} 
	  				\item[$\bullet$] Matière
					\item[$\bullet$] Classe
					\item[$\bullet$] Prof
					\item[$\bullet$] Salle
					\item[$\bullet$] Jour
					\item[$\bullet$] Heure début
					\item[$\bullet$] Heure fin
				\end{itemize} 
			\end{minipage} 
		};

	

		\end{tikzpicture}

		\paragraph{}
		Une fois que la répartition des classes avec leurs matières et leurs professeurs nous allons pouvoir placer l'ensemble des cours dans notre emploi du temps. Nous allons avoir un Cours qui va être un évènement dans notre emploie du temps. Cet évènement aura une matière, une classe, un prof (ensemble défini dans la première étape), une salle, le jour avec les heures de début et de fin.
	
		\paragraph{}
		Pour chaque matières nous aurons définis au début les salles qui vont être adéquate en place. Nous allons prendre un évènement à placer dans notre emploi du temps et nous allons choisir une salle disponible en fonction de la liste que l'on a avec la matière. Cet évènement va être fixer de semaine en semaine. Du coup la salle sera réserver pour le premier cours que l'on planifier. Pour un autre cours à la même heure et le même jour de la semaine, il ne sera pas possible d'utiliser cette salle. Nous plaçons au fur et à mesure les cours pour répartir les salles entre les différentes classes et matières.
		 

\end{document}