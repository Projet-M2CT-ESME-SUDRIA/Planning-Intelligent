\documentclass[12pt,a4paper,french]{article}

\usepackage[T1]{fontenc}
\usepackage[utf8]{inputenc}
\usepackage{fullpage}
\usepackage[french]{babel} 
\usepackage[official]{eurosym}
\usepackage{txfonts}
\usepackage{graphicx}
\usepackage{lastpage}
\usepackage{fancyhdr}
\usepackage{titlesec}
\usepackage{color}
\usepackage{setspace}
\usepackage[nottoc, notlof, notlot]{tocbibind}
\usepackage{hyperref}
\usepackage[french,ruled,vlined]{algorithm2e}
\usepackage{algorithmic}

\begin{document}

\title{Algorithmes -- Emploi du temps}
\author{Coudray -- Julien -- Tran}
\date{\today}
\maketitle

\section{Pré-traitements}
Avant de réaliser l'emploi du temps, nous procédons à des vérifications sur les données d'entrées afin de détecter toutes les incohérences. Ainsi, nous éliminons au préalable une partie des traitements qui n'aboutiront pas.

\subsection{Le nombre de professeur}
La première vérification concerne le nombre de professeurs en entrée. Nous vérifions s'il y a assez de professeurs pour dispenser les cours de chaque classe.
Ainsi pour un cours donné, l'algorithme somme les disponibilités des professeurs puis compare le résultat au nombre de classe.\\
Si le cours est sur 4 heures, alors la somme des disponibilités est divisée par 2. En effet, le cours en question nécessite deux créneaux consécutifs pour être dispensé.\\

Soit $n$ le nombre de professeurs pouvant donner un cours $c$ et $m$ le nombre de classe devant suivre ce cours.
Pour un cours de 2 heures, nous avons:
\begin{center}
$\sum_{i=0}^n dispo_{prof_i} > m$
\end{center}

Pour un cours de 4 heures, nous avons: 
\begin{center}
$\frac{\sum_{i=0}^n dispo_{prof_i}}{2} > m$
\end{center}

L'opération est répétée pour l'ensemble des cours. 

\begin{algorithm}
\caption{Pré-traitement nombre de professeurs}
\begin{algorithmic}
\FORALL{$Cours$}
\STATE $idCours \leftarrow$ identifiant de $Cours$
\STATE $idPromo \leftarrow$ identifiant de la promotion recevant $Cours$
\STATE $nbClasses \leftarrow$ nombre de classe de la promotion $idPromo$
\FORALL{$Profs$}
\IF {$Profs$ donne le cours $idCours$}
\FORALL {$CreneauxProf$}
\IF {$Profs$ est disponible}
\STATE $nbCreneaux \leftarrow nbCreneaux + 1$
\ENDIF
\ENDFOR
\ENDIF
\ENDFOR
\IF {$Cours$ est sur 4h}
\STATE $nbCreneaux \leftarrow nbCreneaux / 2$
\ENDIF
\IF{$nbClasses > nbCreneaux$}
\STATE display (Erreur sur le nombre de professeur pour la promo $idPromo$)
\STATE EXIT FAILURE
\ENDIF
\ENDFOR
\STATE display (Nombre de professeurs ok)
\end{algorithmic}
\end{algorithm}

\newpage

\subsection{Le nombre de cours total sur le semestre}
La seconde vérification porte sur le nombre d'heures de cours à dispenser à une classe. Ce nombre ne doit pas excéder la totalité des heures du semestre. Le programme somme l'ensemble des cours que possède une classe et le compare au nombre d'heures du semestre.

Soit $n$ le nombre de cours d'une classe $p$, $s$ le nombre de semaines sur un semestre, $c$ le nombre de créneaux sur une semaine et $h$ le nombre d'heures d'un créneau: 

\begin{center}
$\sum_{i=0}^n nbHeures_{cours_i} \leq s*c*h$
\end{center}

\begin{algorithm}
\caption{Pré-traitement nombre d'heures sur le semestre}
\begin{algorithmic}
\FORALL{$Classes$}
\STATE $listCours \leftarrow$ ensemble des cours que suit une classe
\FORALL {$cours$ in $listCours$}
\STATE $nbHours \leftarrow nbHours +$ nombre d'heures du cours $cours$
\ENDFOR
\IF {$nbHours$ > (nombre de semaines du semestre * nombre de créneaux par semaine * nombre d'heures par créneau)}
\STATE display(Erreur, trop d'heures pour la classe $Classes$)
\STATE EXIT FAILURE
\ENDIF
\ENDFOR
\STATE display(Nombre d'heures de cours ok)
\end{algorithmic}
\end{algorithm}

Une fois ces pré-traitements réalisés, nous pouvons commencer la conception de l'emploi du temps de l'école.

\newpage
\section{Réalisation de l'emploi du temps}
La réalisation de l'emploi du temps de l'école est répété durant un temps défini. Mais pourra être interrompu si nous parvenons à trouver un emploi du temps planifiant l'intégralité des cours de l'école.\\
La réalisation d'un l'emploi du temps se déroule en deux étapes. Tout d'abord, chaque classe d'une promotion suit les mêmes cours avec la même liste d'enseignants potentiels. Nous commençons par répartir les cours en indiquant la semaine où le cours commence et celle où il se termine. 
Enfin, un emploi du temps final sera réalisé semaine par semaine avec chaque cours placés sur ses créneaux respectifs. Chaque cours sera placés aléatoirement sur un créneau possible.

L'emploi du temps ayant le plus de cours placé sera l'emploi du temps final.

\begin{algorithm}
\caption{Principe général de conception des emplois du temps}
\begin{algorithmic}
\FORALL{$Promo$}
\STATE $idCours \leftarrow$ liste de tous les cours que doivent suivre la promotion $Promo$
\STATE $programmeSemestre \leftarrow $ repartitionCoursSemestre($idCours$)
\STATE repartitionCoursPromotions($Promo, programmeSemestre$)
\ENDFOR
\end{algorithmic}
\end{algorithm}

\subsection{Répartition du programme sur le semestre}
La première étape de l'algorithme de résolution est de répartir de l'ensemble du programme de chaque promotion sur le semestre. Cette étape consiste à indiquer pour chaque cours la date de début et de fin semaine.\\

La répartition se déroule en deux étapes : 
\begin{itemize}
\item Le trie des cours 
\item Le placement des cours sur le semestre\\
\end{itemize}

L'objectif est de répartir au mieux les cours sur le semestre. Il faut donc réussir à placer le maximum de cours les uns à la suite des autres. C'est pourquoi nous plaçons les cours les plus longs en premier, puis nous vérifions s'il est possible de placer un nouveau cours derrière ceux-là, sinon nous le plaçons en début de semestre.\\

Un cours de 4 heures impose plus de contraintes. En effet, il s'agit d'un cours où le professeur et la classe doivent avoir deux créneaux consécutifs dans la même demi-journée. C'est pourquoi un cours de 4 heures doit être planifié sur le semestre avant un cours de 2 heures.\\

Pour se faire, les cours vont être séparés en deux listes : une pour les cours de 4 heures et une autre pour les cours de 2 heures. Ainsi pour chacunes des listes, un trie décroissant est effectué par rapport au nombre de semaines sur lequel les cours vont être suivis.\\

$semaineDebut_{coursPlace} + nbSemaine_{coursPlace} + nbSemaine_{nouveauCours} \leq nbSemaine_{semestre}$\\

Chaque élément du semestre va avoir les informations suivantes :
\begin{itemize}
\item L'identifiant du cours
\item Le numéro du début de la semaine
\item Le nombre de semaine du cours
\item Le cours qui le suit
\end{itemize}

\begin{algorithm}
\caption{Algorithme principale de la répartition des cours sur le semestre}
\begin{algorithmic}
\REQUIRE liste $idCours$, liste $programmeSemestre$
\FORALL {$idCours$}
\IF {$idCours$ est un cours sur 2h}
\STATE $idCours2 \leftarrow$ pushback $idCours$
\ELSE
\STATE $idCours4 \leftarrow$ pushback $idCours$
\ENDIF
\ENDFOR
\STATE Trie de $idCours2$ par nombre de semaine de cours décroissant
\STATE Trie de $idCours4$ par nombre de semaine de cours décroissant
\STATE $idCours$ est vider
\FORALL {$idCours4$}
\STATE $idCours \leftarrow$ pushback $idCours4$
\ENDFOR
\FORALL {$idCours2$}
\STATE $idCours \leftarrow$ pushback $idCours2$
\ENDFOR
\RETURN $programmeSemestre \leftarrow$ repartitionDesCours($idCours$)
\end{algorithmic}
\end{algorithm}

\begin{algorithm}
\caption{repartitionDesCours($idCours$)}
\begin {algorithmic}
\REQUIRE liste $idCours$ triée par nombre de semaine d'un cours et par cours de 4H et 2H
\STATE initialisation de $programmeSemestre$
\FORALL {$idCours$}
\FORALL {$cours$ in $programmeSemestre$}
\IF {$cours$ a été placé}
\STATE checkNextCourse($idCourses, cours$)
\IF {$idCours$ a été programmé}
\STATE $coursPlace \leftarrow true$
\STATE BREAK
\ENDIF
\ENDIF
\ENDFOR
\IF {$coursPlace == false$}
\STATE $programmeSemester \leftarrow$ pushback $idCours$ en le configurant en début de semestre
\ENDIF
\ENDFOR
\RETURN $programmeSemestre$
\end{algorithmic}
\end{algorithm}


\newpage

\begin{algorithm}
\caption{checkNextCourse($idCours, cours$)}
\begin {algorithmic}
\IF {$cours$ a un cours après lui}
\STATE checkNextCourse($idCourses, cours$ du $cours$ suivant)
\ELSIF {$semaineDebut_{coursProgrammes} + nbSemaine_{coursProgramme} + nbSemaine_{idCourses} \leq nbSemaine_{semestre}$}
\STATE $programmeSemestre \leftarrow$ pushBack $idCours$ en le configurant après le cours $cours$
\ENDIF
\end{algorithmic}
\end{algorithm}

Après avoir réaliser cet emploi du temps, nous pouvons commencer à placer les cours sur les créneaux des classes concernées

\subsection{Répartion des cours sur leurs créneaux}

A partir de l'emploi du temps du semestre, nous allons planifier les cours sur les créneaux des classes. A la fin de chaque semaine, les classes d'une même promotion doivent être au même point du programme. Ainsi, l'emploi du temps est réalisé en parallèle pour chaque promotion semaine par semaine.

Pour chacune des semaines, nous allons récupérer la liste des cours à dispenser depuis l'emploi du temps semestriel. A partir des semaines de début et de fin de cours, nous en déduisons s'il doit être donné sur cette semaine.

Avec ce programme, nous allons pouvoir commencer à placer les cours sur les différents créneaux des classes. Cela va être fait en trois étapes : \\
\begin{itemize}
\item Si le cours a déjà été placé à la semaine précédente
\item La récupération de la liste des professeurs pouvant enseigner les cours de la semaine
\item Le placement des nouveaux cours du semestre\\
\end{itemize}

Dans le cas où un cours n'a pu être placé faute de créneaux disponibles, nous avons décidé de le mettre dans une liste contenant l'ensemble des cours non placés et de ne pas tenter de le replacer sur les semaines suivantes. Ainsi cette liste contiendra l'identifiant du cours, l'identifiant du professeur et les semaines où le cours n'a pu être placé.

\begin{algorithm}
\caption {Algorithme principal de la répartition des cours sur les créneaux des classes}
\begin{algorithmic}
\REQUIRE le programme du semaine $prog$
\FORALL{$semaine$ du semestre}
\STATE $progSemaine \leftarrow$ getProgrammeSemaine($prog, semaine$)
\STATE placementAncienCours($progSemaine, listeClasses, semaine$)
\IF {il y a des cours à placer encore dans la semaine}
\STATE $profSemaine \leftarrow$ getProfSemaine($progSemaine$)
\STATE placementNouveauCours($listeCalsses, progSemaine, profSemaine, semaine$)
\ENDIF
\IF {une erreur est survenu dans la réalisation du planning}
\RETURN 0
\ENDIF
\ENDFOR
\RETURN 1
\end{algorithmic}
\end{algorithm}

\begin{algorithm}
\caption{Méthode pour récupérer le programme d'une semaine}
\begin{algorithmic}
\REQUIRE le programme du semaine $prog$ et la semaine du semestre $semaine$
\FORALL {$cours$ du programme}
\IF {semaineDebut de $cours \leq semaine \AND $ semaine fin de $cours > semaine$}
\STATE $progSemaine \leftarrow$ pushback $cours$
\ENDIF
\ENDFOR
\RETURN $progSemaine$
\end{algorithmic}
\end{algorithm}

\subsubsection{Placement de cours déjà fixé la semaine précédente}

Après avoir récupéré le programme de la semaine, nous vérifions si l'un des cours a déjà été placé la semaine précédente. Si c'est le cas, nous allons vérifier que le professeur ayant donné le cours est toujours disponible sur le créneau et nous plaçons le cours.\\

Dans le cas où le cours a pu être redonner pour toutes les classes de la promotion, nous pouvons supprimer le cours dans le programme de la semaine. Sinon, soit il s'agit d'un nouveau cours, soit toutes les classes ne l'ont pas reçu. Ce dernier cas arrive lorsqu'un professeur n'est plus disponible ou quand le cours n'a pu être placé pour toutes les classes de la promotion.\\

Ceci va permettre à une classe d'avoir le même cours sur le même créneaux avec le même professeur semaine après semaine.

\begin{algorithm}
\caption{Méthode pour placer les cours précédemment planifier}
\begin{algorithmic}
\REQUIRE le programme de la semaine $prog$, la liste des classes $classes$, la semaine du semestre $semaine$
\STATE $nbCourseAjout \leftarrow 0$
\STATE $nouveauCours \leftarrow false$
\IF {La première semaine à déjà été planifié}
\FORALL {$cours$ du programme de la semaine}
\STATE coursDejaProgrammeAvant($cours, classes, nbCoursAjout, nouveauCours$)
\IF {$nbCoursAjout = $nombre $classes$}
\STATE $coursASupprimer \leftarrow$ pushback $cours$
\ELSE 
\STATE $nouveauCours \leftarrow faux$
\ENDIF
\STATE $nbCourseAjout \leftarrow 0$
\ENDFOR
\FORALL {$coursASupprimer$}
\STATE $progSemestre \leftarrow$ supprimer $progSemestre(coursASupprimer)$
\ENDFOR
\ENDIF
\end{algorithmic}
\end{algorithm}


\begin{algorithm}
\caption{Méthode pour savoir si un cours a déjà été programmé avant}
\begin{algorithmic}
\REQUIRE le cours de la semaine $cours$, la liste des classe $classes$ , la semaine du semestre $semaine$
\FORALL {$classes$}
\IF {$classes$ a reçu le cours la semaine $semaine -1$}
\STATE ajoutDuCours($classes, cours, semaine$)
\STATE $nbCoursAjoute \leftarrow nbCoursAjoute + 1$
\ENDIF
\ENDFOR
\IF {$nbCoursAjoute = $nombre de $classes$}
\STATE $nouveauCours \leftarrow faux$
\ELSE
\STATE $nouveauCours \leftarrow vrai$ 
\ENDIF
\end{algorithmic}
\end{algorithm}


\begin{algorithm}
\caption{Méthode pour ajouter le cours par rapport à la semaine d'avant}
\begin{algorithmic}
\REQUIRE la classe $classe$, la matière $cours$, la semaine du semestre $semaine$
\STATE $idProf \leftarrow $identifiant du professeur donnant $cours$ la $semaine - 1$
\STATE $creneau \leftarrow $créneau de $cours$ la $semaine - 1$
\IF {$cours$ est sur 4 heures}
\IF {$prof$ est disponible à $semaine, creneau \AND prof$ est disponible $semaine, creneau + 1 \AND classe$ est disponible à $semaine, creneau \AND classe$ est disponible $semaine, creneau +1$}
\STATE planification $cours$ avec $prof$ sur $semaine$ et $creneau$
\STATE planification $cours$ avec $prof$ sur $semaine$ et $creneau +1$
\ENDIF
\ELSE
\IF {$prof$ est disponible à $semaine, creneau \AND$ $prof$ est disponible $semaine, creneau + 1$}
\STATE planification $cours$ avec $prof$ sur $semaine$ et $creneau$
\ENDIF
\ENDIF
\end{algorithmic}
\end{algorithm}

\newpage

\subsubsection{Planification des nouveaux cours du semestre}

A partir du programme du semestre, nous récupérons l'ensemble des professeurs pouvant enseigner la liste des matières. Tous les professeurs ne se verront pas forcément attribuer un cours car plusieurs professeurs peuvent enseigner le même cours.\\

Pour tous les cours restant à planifier, nous allons à chaque fois trouver le couple promotion-professeur ayant le moins de créneaux en communs. En effet si nous plaçons des couples ayant plus de disponibilités avant un couple qui en a moins, il pourrait bloquer l'ensemble des disponibilités de ce dernier.\\

Ensuite, nous sélectionnons un créneaux aléatoirement parmi les choix possibles pour placer un cours. Nous réalisons un tirage aléatoire pour pouvoir essayer plusieurs combinaisons.\\

Dans le cas d'un cours de 4 heures, il se peut que nous n'ayons aucun créneau permettant de mettre les 4 heures à la suite. Dans ce cas nous mettons le cours dans la liste des cours n'ayant pu être planifié. 
Auquel cas, nous plaçons le cours de la classe sur le créneau en modifiant les disponibilités du professeurs. 

\begin{algorithm}
\caption {Méthode pour ajouter un nouveau cours}
\begin{algorithmic}
\REQUIRE $progSemaine, profSemaine, listClasses$
\STATE $nbCours \leftarrow $ nombre de cours dans $progSemaine * $ nombre de classe dans $listClasses$
\FOR {i := 0  \TO nbCours }
\STATE meilleurConnexion($progSemaine, profSemaine, listClasses, semaine$)
\IF {on trouve une connexion}
\STATE ajoutCours($progSemaine, profAAjouter, classesAAjouter, semaine$)
\ELSE
\RETURN 0
\ENDIF
\ENDFOR
\RETURN 1
\end{algorithmic}
\end{algorithm}

\newpage

\begin{algorithm}
\caption {Méthode pour trouver la plus meilleur connexion}
\begin{algorithmic}
\REQUIRE $progSemaine, profSemaine, listClasses, semaine$
\STATE $buf \leftarrow 23$
\FORALL {$profSemaine$}
\FORALL {$listClasses$}
\STATE $nbConnections \leftarrow $ nbCreneauCommun($profSemaine, listClasses, semaine$)
\IF {$nbConnections > 0 \AND nbConnections < buf$}
\STATE $buf \leftarrow nbConnections$
\STATE $profAAjouter \leftarrow profSemaine$
\STATE $promoAAjouter \leftarrow listClasses$
\ENDIF
\ENDFOR
\ENDFOR
\end{algorithmic}
\end{algorithm}

\begin{algorithm}
\caption {Méthode pour compter le nombre de connection}
\begin{algorithmic}
\REQUIRE $prof, classe, semaine, progSemaine$)
\FORALL {$cours$ donnés par $prof$}
\IF {$promo$ doit recevoir $cours$ sur $semaine$ \AND $cours$ n'a pas encore été placé pour $promo$ sur $semaine$}
\STATE $coursPossible \leftarrow vrai$
\STATE BREAK
\ELSE
\STATE $coursPossible \leftarrow faux$
\ENDIF
\ENDFOR
\IF {$coursPossible$}
\RETURN $nbConnection \leftarrow $ somme des disponibilité commune de $prof$ et $promo$
\ENDIF
\RETURN -1
\end{algorithmic}
\end{algorithm}

\newpage

\begin{algorithm}
\caption {Méthode pour ajouter un cours à une classe}
\begin{algorithmic}
\REQUIRE $progSemaine, profAAjouter, classesAAjouter, semaine$
\FORALL {$cours$ de $profAAjouter$}
\IF {$promo$ doit recevoir $cours$ sur $semaine$ \AND $cours$ n'a pas encore été placé pour $promo$ sur $semaine$}
\STATE creationCours($prof, promo, cours, semaine$)
\STATE BREAK
\ENDIF
\ENDFOR
\end{algorithmic}
\end{algorithm}

\newpage

\begin{algorithm}
\caption {Méthode pour créer le cours à la classe}
\begin{algorithmic}
\IF {$cours$ n'a pas été programmé à $semaine -1$ pour $classes$}
	\STATE ajout de $cours$ dans la liste des cours non planifié
\ELSIF {$cours$ est sur 4 heures}
	\FORALL {$creneau$}
		\IF {$classe$ est libre à $semaine, creneau$ \AND $classe$ est libre à $semaine, creneau + 1$ \AND $prof$ est libre à 			$semaine, creneau$  \AND $prof$ est libre à $semaine, creneau+1$ }
			\STATE $creneauxPossibles \leftarrow$ pushback $creneau$
		\ENDIF
	\ENDFOR
	\IF {$creneauxPossibles$ n'est pas vide}
		\STATE $creneau \leftarrow$ choix aléatoire dans $creneauxPossibles$
		\STATE mise en place du cours et des données ($cours, classe, prof, semaine, creneau$)
		\STATE mise en place du cours et des données ($cours, classe, prof, semaine, creneau +1$)
	\ENDIF
	\STATE ajout de $cours$ dans la liste des cours non planifié
\ELSE
	\FORALL {$creneau$}
		\IF {$classe$ est libre à $semaine, creneau$ \AND $prof$ est libre à $semaine, creneau$}
			\STATE $creneauxPossibles \leftarrow$ pushback $creneau$
		\ENDIF
	\STATE $creneau \leftarrow$ choix aléatoire dans $creneauxPossibles$
	\STATE mise en place du cours et des données ($cours, classe, prof, semaine, creneau$)
	\ENDFOR
\ENDIF
\end{algorithmic}
\end{algorithm}






\section{Déplacements des cours}

Une fois l'emploi du temps réalisé, nous avons la possibilité de déplacer des cours et/ou de placer manuellement les cours qui n'ont pu être placés lors de l'exécution du programme.

\subsection {Déplacement d'un cours existant}

Dans un premier temps, il faut choisir la classe pour laquelle nous souhaitons effectuer le changement. Ensuite nous sélectionnons le cours à déplacer, et la liste des créneaux communs au professeur et à la classe apparait. En sélectionnant le créneau dans lequel nous souhaitons déplacer le cours, le changement s'effectue directement en mettant à jour les données relatives à la plannification.

\subsection {Ajout d'un cours non placé}

La liste des cours non placés apparait. Il n'y a qu'à sélectionner le cours que nous voulons placer, et la liste des créneaux sur lequel il peut être placé apparait. Les instructions suivantes sont les mêmes que pour le déplacement d'un cours.

\end{document}