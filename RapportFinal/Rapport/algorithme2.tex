\subsection{Répartion des cours sur leurs créneaux}

Grâce à la répartition des matières sur le semestre, nous allons savoir quels sont les cours à organiser sur chaque semaine. Cette répartition a été effectuée avec un objectif majeur : maintenir un emploi du temps homogène de manière à ce que chaque semaine, toutes les classes d'une même promotion aient reçus les mêmes cours.

Afin d'obtenir le meilleur emploi du temps possible, nous allons mettre en place un système de répartition aléatoire des cours. Le programme va réitérer cette fonction un certain nombre de fois, le nombre d'itération étant un paramètre modifiable. Le programme arrête d'itérer lorsqu'il trouve une solution parfaite (tous les cours placés correctement) ou lorsqu'il atteint le nombre d'itérations maximum. Il va alors générer les emplois du temps à partir de la solution retenue.

Cette fonction va récupérer pour chaque semaine la liste des cours à dispenser, et procéder au placement des cours, qui s'articule en deux étapes :\\

\begin{itemize}
\item Copie des cours placés la semaine précédente et qui doivent encore l'être sur cette semaine
\item Placement des nouveaux cours de la semaine après récupération des professeurs qualifiés pour les enseigner\\
\end{itemize}

Lorsqu'un cours ne peut être placé faute de disponibilité, il est ajouté à une liste dédiée aux cours non placés, et est retiré de la liste des cours à placer. Cette liste permet d'établir la qualité de la solution : plus elle est remplie, et plus la solution est mauvaise. Cette liste contiendra l'identifiant de la matière, l'identifiant du professeur et les semaines où le cours aurait dû être placé.

\newpage

\begin{algorithm}
\caption {Algorithme principal de la répartition des cours sur les créneaux des classes}
\begin{algorithmic}
\REQUIRE le programme de la semaine $prog$
\FORALL{$numSemaine$ du semestre}
\STATE $progSemaine \leftarrow$ getProgrammeSemaine($prog, numSemaine$)
\STATE placementAncienCours($progSemaine, listeClasses, numSemaine$)
\IF {il y a des nouveaux cours à placer dans la semaine}
\STATE $profSemaine \leftarrow$ getProfSemaine($progSemaine$)
\STATE placementNouveauCours($listeClasses, progSemaine, profSemaine, numSemaine$)
\ENDIF
\IF {une erreur est survenu dans la réalisation du planning}
\RETURN 0
\ENDIF
\ENDFOR
\RETURN 1
\end{algorithmic}
\end{algorithm}

\begin{algorithm}
\caption{Méthode pour récupérer le programme d'une semaine}
\begin{algorithmic}
\REQUIRE le programme de la semaine $prog$ et la semaine du semestre $numSemaine$
\FORALL {$cours$ du programme}
\IF {semaineDebut de $cours \leq numSemaine \AND $ semaine fin de $cours > numSemaine$}
\STATE $progSemaine \leftarrow$ pushback $cours$
\ENDIF
\ENDFOR
\RETURN $progSemaine$
\end{algorithmic}
\end{algorithm}

\subsubsection{Placement de cours déjà fixé la semaine précédente}

Lorsqu'on récupère le programme d'une semaine, la première étape consiste à vérifier si certains cours ont déjà été placés la semaine précédente, et le cas échéant à les copier dans l'emploi du temps de cette semaine, en vérifiant bien que le professeur est toujours disponible. Le cours sera ajouté à la liste des cours non placés si le professeur n'est plus disponible. Lorsqu'un cours a été placé pour toutes les classes d'une promotion, nous le supprimons de la liste des cours à dispenser. Dans le cas contraire, le cours est ajouté à la liste des cours non placés. Ceci va permettre à une classe d'avoir le même cours sur le même créneau avec le même professeur d'une semaine à l'autre.

\begin{algorithm}
\caption{Méthode pour placer les cours précédemment planifiés}
\begin{algorithmic}
\REQUIRE le programme $prog$ de la semaine, la liste des classes $classes$, la semaine $numSemaine$
\STATE $nbCourseAjout \leftarrow 0$
\STATE $nouveauCours \leftarrow false$
\IF {La première semaine a déjà été planifiée}
\FORALL {$cours$ du programme de la semaine}
\STATE coursDejaProgrammeAvant($cours, classes, nbCoursAjout, nouveauCours$)
\IF {$nbCoursAjout = $nombre $classes$}
\STATE $coursASupprimer \leftarrow$ pushback $cours$
\ELSE 
\STATE $nouveauCours \leftarrow faux$
\ENDIF
\STATE $nbCourseAjout \leftarrow 0$
\ENDFOR
\FORALL {$coursASupprimer$}
\STATE $progSemestre \leftarrow$ supprimer $progSemestre(coursASupprimer)$
\ENDFOR
\ENDIF
\end{algorithmic}
\end{algorithm}

\begin{algorithm}
\caption{Méthode pour savoir si un cours a déjà été programmé avant}
\begin{algorithmic}
\REQUIRE le cours de la semaine $cours$, la liste des classe $classes$ , la semaine du semestre $numSemaine$
\FORALL {$classes$}
\IF {$classes$ a reçu le cours la semaine $numSemaine -1$}
\STATE ajoutDuCours($classes, cours, numSemaine$)
\STATE $nbCoursAjoute \leftarrow nbCoursAjoute + 1$
\ENDIF
\ENDFOR
\IF {$nbCoursAjoute = $nombre de $classes$}
\STATE $nouveauCours \leftarrow faux$
\ELSE
\STATE $nouveauCours \leftarrow vrai$ 
\ENDIF
\end{algorithmic}
\end{algorithm}


\begin{algorithm}
\caption{Méthode pour ajouter le cours par rapport à la semaine d'avant}
\begin{algorithmic}
\REQUIRE la classe $classe$, la matière $cours$, la semaine du semestre $numSemaine$
\STATE $idProf \leftarrow $identifiant du professeur donnant $cours$ la $numSemaine - 1$
\STATE $creneau \leftarrow $créneau de $cours$ la $numSemaine - 1$
\IF {$cours$ est sur 4 heures}
\IF {$prof$ est disponible à $numSemaine, creneau \AND prof$ est disponible $numSemaine, creneau + 1 \AND classe$ est disponible à $numSemaine, creneau \AND classe$ est disponible $numSemaine, creneau +1$}
\STATE planification $cours$ avec $prof$ sur $numSemaine$ et $creneau$
\STATE planification $cours$ avec $prof$ sur $numSemaine$ et $creneau +1$
\ENDIF
\ELSE
\IF {$prof$ est disponible à $numSemaine, creneau \AND$ $prof$ est disponible $numSemaine, creneau + 1$}
\STATE planification $cours$ avec $prof$ sur $numSemaine$ et $creneau$
\ENDIF
\ENDIF
\end{algorithmic}
\end{algorithm}

\newpage

\subsubsection{Planification des nouveaux cours du semestre}

Une fois que tous les cours de la semaine précédente ont été traités, nous pouvons nous occuper des nouveaux cours. Nous disposons de la liste des cours de la semaine dans laquelle il ne reste que les nouveaux cours. Nous commençons par récupérer l'ensemble des professeurs capables d'enseigner ces matières.\\

Nous procédons alors à la sélection du couple classe-professeur ayant le moins de créneaux en commun. Ce couple sera prioritaire sur les autres car c'est celui qui dispose du moins de créneau et qui est par conséquent le plus contraignant.\\

C'est maintenant qu'intervient le facteur aléatoire : nous allons répertorier les créneaux communs au professeur et à la classe, et en choisir un au hasard.\\

Lorsqu'un cours n'a pas pu être ajouté à une classe, il est ajouté à la liste des cours non placés.

Dans le cas d'un cours de 4 heures, il se peut que nous n'ayons aucun créneau permettant de mettre les 4 heures à la suite. Dans ce cas nous mettons le cours dans la liste des cours n'ayant pu être planifiés. 

\begin{algorithm}
\caption {Méthode pour ajouter un nouveau cours}
\begin{algorithmic}
\REQUIRE $progSemaine, profSemaine, listClasses$
\STATE $nbCours \leftarrow $ nombre de cours dans $progSemaine * $ nombre de classes dans $listClasses$
\FOR {i := 0  \TO nbCours }
\STATE meilleureConnexion($progSemaine, profSemaine, listClasses, numSemaine$)
\IF {on trouve une connexion}
\STATE ajoutCours($progSemaine, profAAjouter, classesAAjouter, numSemaine$)
\ELSE
\RETURN 0
\ENDIF
\ENDFOR
\RETURN 1
\end{algorithmic}
\end{algorithm}

\newpage

\begin{algorithm}
\caption {Méthode pour trouver la meilleure connexion}
\begin{algorithmic}
\REQUIRE $progSemaine, profSemaine, listClasses, numSemaine$
\STATE $buf \leftarrow 23$
\FORALL {$profSemaine$}
\FORALL {$listClasses$}
\STATE $nbConnections \leftarrow $ nbCreneauxCommuns($profSemaine, listClasses, numSemaine$)
\IF {$nbConnections > 0 \AND nbConnections < buf$}
\STATE $buf \leftarrow nbConnections$
\STATE $profAAjouter \leftarrow profSemaine$
\STATE $promoAAjouter \leftarrow listClasses$
\ENDIF
\ENDFOR
\ENDFOR
\end{algorithmic}
\end{algorithm}

\begin{algorithm}
\caption {Méthode pour compter le nombre de créneaux communs}
\begin{algorithmic}
\REQUIRE $prof, classe, numSemaine, progSemaine$)
\FORALL {$cours$ donnés par $prof$}
\IF {$promo$ doit recevoir $cours$ sur $numSemaine$ \AND $cours$ n'a pas encore été placé pour $promo$ sur $numSemaine$}
\STATE $coursPossible \leftarrow vrai$
\STATE BREAK
\ELSE
\STATE $coursPossible \leftarrow faux$
\ENDIF
\ENDFOR
\IF {$coursPossible$}
\RETURN $nbConnection \leftarrow $ somme des disponibilités communes de $prof$ et $promo$
\ENDIF
\RETURN -1
\end{algorithmic}
\end{algorithm}

\newpage

\begin{algorithm}
\caption {Méthode pour ajouter un cours à une classe}
\begin{algorithmic}
\REQUIRE $progSemaine, profAAjouter, classesAAjouter, numSemaine$
\FORALL {$cours$ de $profAAjouter$}
\IF {$promo$ doit recevoir $cours$ sur $numSemaine$ \AND $cours$ n'a pas encore été placé pour $promo$ sur $numSemaine$}
\STATE creationCours($prof, promo, cours, numSemaine$)
\STATE BREAK
\ENDIF
\ENDFOR
\end{algorithmic}
\end{algorithm}

\newpage

\begin{algorithm}
\caption {Méthode pour créer le cours à la classe}
\begin{algorithmic}
\IF {$cours$ n'a pas été programmé à $numSemaine - 1$ pour $classes$}
	\STATE ajout de $cours$ dans la liste des cours non planifiés
\ELSIF {$cours$ est sur 4 heures}
	\FORALL {$creneau$}
		\IF {$classe$ est libre à $numSemaine, creneau$ \AND $classe$ est libre à $numSemaine, creneau + 1$ \AND $prof$ est libre à 			$numSemaine, creneau$  \AND $prof$ est libre à $numSemaine, creneau+1$ }
			\STATE $creneauxPossibles \leftarrow$ pushback $creneau$
		\ENDIF
	\ENDFOR
	\IF {$creneauxPossibles$ n'est pas vide}
		\STATE $creneau \leftarrow$ choix aléatoire dans $creneauxPossibles$
		\STATE mise en place du cours et des données ($cours, classe, prof, numSemaine, creneau$)
		\STATE mise en place du cours et des données ($cours, classe, prof, numSemaine, creneau +1$)
	\ENDIF
	\STATE ajout de $cours$ dans la liste des cours non planifiés
\ELSE
	\FORALL {$creneau$}
		\IF {$classe$ est libre à $numSemaine, creneau$ \AND $prof$ est libre à $numSemaine, creneau$}
			\STATE $creneauxPossibles \leftarrow$ pushback $creneau$
		\ENDIF
	\STATE $creneau \leftarrow$ choix aléatoire dans $creneauxPossibles$
	\STATE mise en place du cours et des données ($cours, classe, prof, numSemaine, creneau$)
	\ENDFOR
\ENDIF
\end{algorithmic}
\end{algorithm}





