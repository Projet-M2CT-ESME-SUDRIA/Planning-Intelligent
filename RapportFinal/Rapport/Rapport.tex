\documentclass[12pt,a4paper,french]{article}

\usepackage[T1]{fontenc}
\usepackage[utf8]{inputenc}
\usepackage{fullpage}
\usepackage[french]{babel} 
\usepackage[official]{eurosym}
\usepackage{txfonts}
\usepackage{graphicx}
\usepackage{lastpage}
\usepackage{fancyhdr}
\usepackage{titlesec}
\usepackage{color}
\usepackage{setspace}
\usepackage[nottoc, notlof, notlot]{tocbibind}
\usepackage{hyperref}
\usepackage[french,ruled,vlined]{algorithm2e}
\usepackage{algorithmic}


% Création de la subsubsubsection %
\titleclass{\subsubsubsection}{straight}[\subsection]
\newcounter{subsubsubsection}
\renewcommand\thesubsubsubsection{\thesubsubsection.\arabic{subsubsubsection}}
\renewcommand\theparagraph{\thesubsubsubsection.\arabic{paragraph}}
\titleformat{\subsubsubsection}
  {\normalfont\normalsize\bfseries}{\thesubsubsubsection}{1em}{}
\titlespacing*{\subsubsubsection}
{0pt}{3.25ex plus 1ex minus .2ex}{1.5ex plus .2ex}
\makeatletter
\renewcommand\paragraph{\@startsection{paragraph}{5}{\z@}%
  {3.25ex \@plus1ex \@minus.2ex}%
  {-1em}%
  {\normalfont\normalsize\bfseries}}
\renewcommand\subparagraph{\@startsection{subparagraph}{6}{\parindent}%
  {3.25ex \@plus1ex \@minus .2ex}%
  {-1em}%
  {\normalfont\normalsize\bfseries}}
\def\toclevel@subsubsubsection{4}
\def\toclevel@paragraph{5}
\def\toclevel@paragraph{6}
\def\l@subsubsubsection{\@dottedtocline{4}{7em}{4em}}
\def\l@paragraph{\@dottedtocline{5}{10em}{5em}}
\def\l@subparagraph{\@dottedtocline{6}{14em}{6em}}
\makeatother
\setcounter{secnumdepth}{4}
\setcounter{tocdepth}{4}


\begin{document}

\begin{titlepage}
	\begin{center}
		\begin{spacing}{2}
		{ \Huge{ RÉSOLUTION APPROCHÉE DU PROBLÈME D’ORDONNANCEMENT}}\\[1cm]
		\end{spacing}
		{ \Large{Rapport de projet}}\\[10cm]
		
		\textbf{Auteurs} : 
			Corentin COUDRAY \\
			Christophe JULIEN \\
			Khanh An Noël TRAN \\ 
		\textbf{Groupe} : M2\_9\\
		\textbf{Date} : 03/04/2014\\[2 cm]
	\end{center}	
	% Bas de page %
	\vfill 
	\begin{minipage}[b]{0.3\linewidth}
		\includegraphics[width=30mm]{logo.png}
	\end{minipage}
	\hfill
	\begin{minipage}[b]{0.3\linewidth}
		{\footnotesize{38 rue Molière}}\\
		{\footnotesize{94200 IVRY sur SEINE}}\\
		{\footnotesize{Tel. : 01 56 20 62 00}}\\
		{\footnotesize{Fax. : 01 56 20 62 52}}\\
		{\footnotesize{http://www.esme.fr}}
	\end{minipage}
\end{titlepage}


\renewcommand{\contentsname}{Sommaire}
\tableofcontents
\newpage

\section{Introduction}

La planification sous contraintes est la discipline des mathématiques appliquées consistant à ordonner diverses séquences ou évènements dans un espace-temps limité. Ces planifications impliquent un grand nombre de contraintes, qu'elles soient liées aux disponibilités de personnes, d'emplacements ou à tout autre type de problème.\\

La réalisation d'une planification est longue et périlleuse. Il est très difficile de concilier l'ensemble des contraintes qui accompagnent cet ordonnancement et de les organiser de manière optimale. La mise en place d'un calendrier est souvent faite à la main ou de manière approximative avec d'autres applications.\\

L'objectif de ce projet est donc de réaliser une application, permettant de générer une planification tenant compte de toutes les contraintes que l'on aura indiquées au préalable. 

\newpage
\section{Contexte}

Ce sujet a été proposé dans le but d'alléger le travail de M. Maeso, responsable administratif des études. En effet, chaque année, il doit créer manuellement l'intégralité des emplois du temps de l'école, en tenant compte de toutes les contraintes de disponibilités des professeurs, des salles, des élèves et des matières. L'application pourra lui faire gagner un temps non-négligeable dans son travail.

\newpage
\section{Revue de littérature}

Aujourd'hui, il existe déjà des systèmes de planification informatique, mais ils correspondent toujours à des cas bien spécifiques : organisation du personnel médical dans un hôpital, organisation du personnel dans une base militaire, etc. Mais il semble impossible à l'heure actuelle d'imaginer un logiciel générique permettant de lister des contraintes et d'élaborer le planning correspondant et adapté quel que soit le milieu professionnel.

Ainsi, il nous faudra concevoir notre solution dans son ensemble. Nous pourrons tout de même nous inspirer des méthodes de résolution de ces plannings pour essayer de les adapter à notre problème.

En l'occurrence, nous avons trouvé quelques pistes de recherche :
\begin{itemize}
\item Algorithme des colonies de fourmis : cet algorithme est basé sur le comportement réel des fourmis, qui parviennent, par le biais de rétroactions, à trouver le chemin le plus court entre une source de nourriture et leur nid.
\cite{02-algoFourmis}
\item Recherche tabou : cet algorithme consiste à partir d'une position donnée, à explorer le voisinage pour choisir la position qui minimise la fonction "objectif". Le mot "tabou" vient du fait qu'une fois une position écartée, nous ne pouvons plus y revenir. Cet algorithme procède donc par élimination jusqu'à trouver une solution convenable.
\cite{03-algoTabou}
\end{itemize}

\newpage
\section{Cahier des charges}
\subsection{Mise en contexte du problème}

Les algorithmes nécessaires à l'élaboration d'un tel projet relèvent des problèmes dit "NP- complets". Les problèmes NP-complets sont des problèmes qui sont au minimum aussi difficiles que les problèmes NP. Ceux-ci sont des problèmes de décision à temps polynomial.

Cela signifie qu'il n'existe pas de solution exacte, et qu'on ne peut les résoudre avec les méthodes de résolution classiques.

Même s'il n'existe pas de solution déterministe, nous pouvons malgré tout arriver à une solution "acceptable", c'est à dire répondant à l'ensemble des exigences du problème sans pour autant être optimale dans son organisation, en mettant en oeuvre des algorithmes de résolution rétroactifs.

Le temps de résolution de ces problèmes augmente de manière exponentielle par rapport à l'augmentation des paramètres d'entrée. Nous devrons toujours nous contenter d'une solution qui sera au mieux "satisfaisante".
\cite{01-npComplet}

\subsection{Langage utilisé}
Nous avons décidé de programmer ce projet en C++ car c'est un langage objet qui permet de bien organiser son code, notamment grâce au système de classes, au polymorphisme, et l'héritage, etc. De plus, il s'agit d'un langage que nous avons beaucoup pratiqué durant notre cursus.
Le Java répond également à ces critères, mais il est moins rapide, et un langage rapide est primordial pour une résolution raisonnable en temps de notre problème.

\subsection{Interface graphique}
Pour concentrer nos efforts sur la résolution du problème, nous avons également décidé de ne pas inclure l'interface graphique à notre projet, pour des raisons de temps. Cependant, nous pourrons mettre en place une petite interface utilisateur pour le programme qui gèrera les données.

\subsection{Les données d'entrée du problème}
Les données d'entrée du problème sont stockées dans des fichiers externes. En effet, la nature et la quantité des données ne nécessitent pas de base de données. Cette méthode rend facile la vérification de l'exactitude des données saisies lors de l'exécution du programme. Ainsi il n'est pas nécessaire de relancer le programme lorsque nous voulons modifier des données.

Ce système permet également de mettre en place un archivage des données : l'utilisateur n'aura qu'à copier-coller les fichiers dans un dossier séparé.

\subsection{Les données de sortie du problème}
En sortie, chaque classe, professeur et salle auront leur emploi du temps, stockés dans des fichiers externes.

\subsection{Modélisation des données}
\subsubsection{Les professeurs}
Un professeur sera défini par :
\begin{itemize}
\item son nom
\item ses disponibilités
\item la liste des cours qu'il peut enseigner
\end{itemize}

\subsubsection{Les classes}
Une classe sera définie par:
\begin{itemize}
\item son nom
\item le nombre d'élèves
\item la liste des cours au programme
\end{itemize}

\subsubsection{Les cours}
Un cours sera défini par:
\begin{itemize}
\item son nom
\item le nombre d'heures de cours
\item le type de salle dans laquelle il doit être donné (un type de cours)
\end{itemize}

\subsubsection{Les salles}
Une salle sera définie par:
\begin{itemize}
\item son nom
\item le nombre de places
\item le type : salle de cours, salle de TP
\end{itemize}

\newpage
\section{Représentation formelle}

\newtheorem{definition}{Définition}
\newtheorem{lemme}{Lemme}
\newtheorem{proposition}{Proposition}
\newtheorem{theoreme}{ThŽorme}
\newtheorem{eg}{Exemple}
\newcommand{\ds}{\displaystyle}



Dans l'ensemble de notre travail, un certain nombre de concepts mathématiques apparaîtront de manière récurrente. Cette section rappelle de façon formelle ces principes.\\

Soit $\mathbb{E}$ un ensemble et  $\mathcal{R}$ une relation binaire sur les éléments de $\mathbb{E}$. 

\begin{definition}{\emph{Ordre strict : }}
$\mathbb{E}$ muni de $\mathcal{R}$ est une structure d'ordre strict si et seulement si $\mathcal{R}$ est antisymétrique, non réflexive et transitive.\\
\end{definition}

\begin{definition}{\emph{Ordre partiel :} }
$\mathbb{E}$ muni de $\mathcal{R}$ est une structure d'ordre partiel quand une partie seulement des éléments de $\mathbb{E}$ sont soumis à une une relation d'ordre strict.\\
\end{definition}

\begin{definition}{\emph{Préordre :} }
$\mathbb{E}$ muni de $\mathcal{R}$ est une structure de pré-ordre si et seulement si $\mathcal{R}$  est une relation binaire réflexive et transitive.\\
\end{definition}



$(\mathcal{X},\preceq)$ ordre partiel. \\
$(\bigcup\mathcal{X},\prec)$ ordre strict. \\


Le problème de l'organisation l'emploi du temps est la donnée de : 
\begin{enumerate}
\item Un ensemble de matières (une promotion) $\mathcal{X}:=\{X_{1},\dots, X_{n}\}$ muni d'un ordre partiel $\mathcal{O}$. \item Un ensemble de blocs ordonnés (ordre strict) $W$ correspondant à un ensemble de semaines de cours. Nous sommes convenus que $|W|:=k22$.\footnote{Considérant qu'un bloc est composé de quatre heures consécutives et considérant qu'une semaine est une juxtaposition de 22 blocs.}
\item Un élément $X_{i_{i\in\overline{1,n}}}\in\mathcal{X}$ est un ensemble de cours muni d'un ordre strict sur des blocs de quatre heures quelle que soit la matière .i.e. quels que soient $i$, $i'$, deux indices de matière, et $j$, $j'$ deux indices de cours, nous avons $x_{i,j} \prec x_{i',j'}$ ou $x_{i,j} \succ x_{i',j'}$


\end{enumerate} 

\paragraph{Tâche : } Déterminer un ordre des $x_{i,j}$ ($i$ correspond à un indice de matière, $j$ un indice de cours de la matière $X_{i}$) tel que les propriétés suivantes soient respectées :
\begin{itemize}
\item Tous les cours sont placés au sein des $k22$ blocs.
\item l'ordre partiel sur les matières est respecté
\item l'ordre partiel sur les cours est respecté
\item un cours ne peut avoir lieu plus d'une fois une même semaine quelle que soit la matière.
\item un cours occupe un même bloc sur l'ensemble des $k22$ blocs.
\end{itemize}   

Nous introduisons un niveau de conflit compris entre $1$ et $n$. Il représente le nombre de professeurs disponibles pour enseigner une matière donnée\footnote{ce niveau de conflit peut vraisemblablement concerner les salles aussi}.

\newpage
\section{Les acteurs}        
Il existe deux principaux types d'acteur pour notre logiciel:

\subsection{Premier acteur : Saisie des données}
Le rôle du premier acteur est de saisir toutes les données relatives aux professeurs, salles, et élèves, et tous les paramètres nécessaires à la génération de l'emploi du temps.
        
\subsection{Second acteur : Maintenance de l'emploi du temps}
Etablir un emploi du temps est une bonne chose, mais il faut  également pouvoir le modifier au cours de l'année si des évènements imprévus doivent être rajoutés.
Le rôle du deuxième acteur est donc de gérer la maintenance de l'emploi du temps au fur et à mesure que l'année avance. Il doit pouvoir rajouter des évènements dans l'emploi du temps des professeurs et/ou des élèves.


\newpage
\section{Cas d'utilisation}
\subsection{Génération de l'emploi du temps}    
\begin{figure}[! ht ]
    \centering
    \begin{minipage}[t]{14 cm}
        \centering
            \fbox{\includegraphics[width=140mm, height=140mm]{CasUtilisation1.png}}
        \caption {Diagramme de cas d'utilisation, génération de l'emploi du temps}
    \end{minipage}
\end{figure}
            
\subsubsection{Objectif}
L'objectif est de créer un emploi du temps à partir de données brutes entrées par l'utilisateur.

\subsubsection{Acteur}
L'acteur est celui qui est chargé de la saisie des données.
        
\subsubsection{Données échangées et description des enchaînements}    
L'acteur en question entre les différentes données propres aux professeurs, classes, salles, et matières. Pour chaque élément, il devra remplir des critères bien spécifiques:

\begin{itemize}
\item Pour chaque professeur, il doit entrer son nom, les matières que celui-ci enseigne et ses disponibilités dans la semaine. 
\item Pour chaque matière, il doit entrer le nom de la matière, le programme qui devra être dispensé au cours de l'année (matières et nombre d'heures).
\item Pour chaque classe, il doit préciser le nom de la classe, la promotion à laquelle elle appartient et le nombre d'élèves qu'elle contient.
\item Pour chaque salle, il doit entrer leur contenance et leur type (salle de TP informatique, salle de cours, etc.)
\end{itemize}
                    
Une fois que toutes les données ont été entrées, le logiciel génère tous les emplois du temps de l'école. 
        
\newpage
\subsection{Maintenance de l'emploi du temps}    
\begin{figure}[! ht ]
    \centering
    \begin{minipage}[t]{14 cm}
        \centering
        \fbox{
            \includegraphics[width=140mm, height=80mm]{CasUtilisation2.png}
        }
        \caption {Diagramme de cas d'utilisation, maintenance de l'emploi du temps}
    \end{minipage}
\end{figure}
            
\subsubsection{Objectif}
Pour une raison ou pour une autre, l'utilisateur peut être amené à rajouter un évènement dans l'emploi du temps. Le logiciel doit donc le permettre en tenant  compte des évènements déjà placés et des contraintes que cela implique, sans avoir à générer à nouveau la totalité de l'emploi du temps. 
                
\subsubsection{Acteur}
Le seul acteur à intervenir dans ce cas est celui chargé de la maintenance.
            
\subsubsection{Données échangées}
Pour placer un cours dans l'emploi du temps, le programme consulte les disponibilités de chaque élément:

\begin{itemize}
\item Les disponibilités des professeurs ou des intervenants
\item Les disponibilités des élèves
\item Les disponibilités d'une salle convenant à l'évènement, et dont la capacité est suffisante pour accueillir tous les élèves.
\end{itemize}
                
\subsubsection{Description des enchaînements}        
\subsubsubsection{Pré-condition}
Lorsque l'utilisateur désire rajouter un évènement, il lui faut connaître les disponibilités de chaque entité, et savoir quelles classes sont concernées par l'évènement, si elles doivent recevoir l'évènement par classe ou par promotion, etc.

\subsubsubsection{Séquence}
L'utilisateur entre les disponibilités de chacun dans le programme, et celui-ci cherche les différents créneaux possibles pour placer le cours. L'utilisateur a alors deux possibilités : soit il laisse le programme placer automatiquement le cours, soit il demande au programme d'afficher la liste des créneaux disponibles afin de placer manuellement le cours. Celui-ci n'a alors qu'à choisir le créneau qui lui semble le plus adapté.

\newpage
\section{Organisation des données}
Nous avons défini la semaine sur 22 créneaux de 2 heures chacun. Avec 2 créneaux le matin et 2 l'après-midi du lundi au samedi matin.

\subsection{Les professeurs}
Les disponibilités du professeur sont représentées par un mot binaire de 22 bits, un bit par créneau. 1 représentant une disponibilité et 0 une indisponibilité.
Le mot binaire d'un professeur correspond à une semaine : chaque nouvelle semaine aura un nouveau mot binaire.

Lorsque nous ajoutons un cours à un professeur, ses disponibilités sur la semaine concernée changent, et il faut donc modifier le mot binaire en conséquence.


\subsection{Les classes}
Une classe comprend une liste de semaines, représentant le semestre ou l'année. Celles-ci possèdent un numéro (ID) définissant leur ordre d'arrivée dans l'année, et 22 créneaux assignés que l'on doit remplir avec les cours planifiés à la classe.

Chaque classe possède également un nombre d'élèves déterminant la taille minimum de la salle dans laquelle se déroulera le cours.

Chaque classe a une liste de cours qu'elle devra obligatoirement suivre.

\subsection{Les matières}
Une matière est définie par son nom, la promotion à laquelle elle est rattachée, et le nombre d'heures à enseigner. Dans le cas d'une matière commune à plusieurs promotions, nous les distinguons par des noms différents : un pour chacune des promotions (ex: Algèbre B1, Algèbre B2).

De la même manière, nous distinguons les cours en classe entière, des cours en demi-groupes et des TPs.
Cela implique un dernier paramètre qui détermine le type de salle dans laquelle la matière doit être enseignée (ex: un TP ne peut pas être fait dans un amphithéâtre).

\subsection{Les salles}
Une salle est définie par son nom, sa capacité (nombre de places), et son type.
La capacité est un paramètre déterminant dans le choix de la salle à attribuer lors du placement d'un cours dans l'emploi du temps.

De même, le type de la salle va de pair avec le type d'une matière explicité précédemment. Ces deux paramètres doivent concorder pour que la salle puisse être attribuée à un cours.

Enfin, il faut également préciser dans quel bâtiment se trouve la salle: dans les locaux d'Ivry-sur-Seine, ou dans ceux de Montparnasse, car comme expliqué précédemment, les professeurs ne peuvent pas se permettre de donner de cours dans les deux locaux lors d'une même demi-journée.

\subsection{Les semaines}
Chaque entité de notre emploi du temps (classes, professeurs, salles) doit être organisée dans le temps. Il faut donc leur attribuer un paramètre 'semaine', correspondant à leur planning d'occupation.
Ces semaines sont mises à jour à chaque fois qu'un nouveau cours sera ajouté au planning.

\subsection{Les fichiers externes}
Les données d'entrée à stocker dans les fichiers sont toutes celles que l'utilisateur devra saisir pour lancer le programme. C'est à dire les données sur les professeurs (nom, disponibilités, matières), sur les matières (nom, nombre d'heures, promotions associées) et les salles (nom, capacité, type, bâtiment).\\

Chaque type de donnée est stockée dans un fichier séparé. Au sein du fichier, une ligne contient tous les paramètres d'un seul élément. Les données seront donc récupérées ligne par ligne à chaque exécution du programme.\\

Les données en sortie du problème, c'est à dire les emplois du temps de chaque éléments (classes, professeurs, salles) seront également dans des fichiers externes.

\subsection{Diagrammes de classes}
Ce diagramme représente les liens entre les différentes classes de notre programme avec les simplifications que nous avons décidé de réaliser dans un premier temps.\\
\newpage
\begin{figure}[! ht ]
    \centering
    \begin{minipage}[t]{14 cm}
        \centering
            \fbox{
                \includegraphics[width=120mm, height=110mm]{DiagrammeClasse1.png}
            }
        \caption {Diagramme de classes de mi-projet}
    \end{minipage}
\end{figure}

\paragraph{}
Le bloc "Teaches" représente les relations de $n$ à $n$ entre 
\begin{itemize}
\item les professeurs("Prof") et les matières("Course") : 
	\begin{itemize}
	\item[$\bullet$] un professeur peut enseigner plusieurs matières
	\item[$\bullet$] une matière peut être enseignée par plusieurs professeurs
	\end{itemize}
\item les promotions ("Promo") et les professeurs :
	\begin{itemize}
	\item[$\bullet$] une promotion a plusieurs professeurs
	\item[$\bullet$] un professeur enseigne à plusieurs promotions
	\end{itemize}
\item les matières et les promotions : 
	\begin{itemize}
	\item[$\bullet$] une promotion a plusieurs matières
	\item[$\bullet$] une matière est enseignée à plusieurs promotions
	\end{itemize}
\end{itemize}

\newpage

\paragraph{}
Ce second diagramme est une ébauche du diagramme des classes final sous réserve de modifications lorsque nous ajouterons les salles à notre problème.\\

\begin{figure}[! ht ]
    \centering
    \begin{minipage}[t]{14 cm}
        \centering
            \fbox{
                \includegraphics[width=120mm, height=110mm]{DiagrammeClasse2.png}
            }
        \caption {Diagramme de classes final envisagé}
    \end{minipage}
\end{figure}

\paragraph{}
Le bloc "Stands" représente les relations de $n$ à $n$ entre les matières ("Course") et les salles de cours ("Room"):
\begin{itemize}
\item une matière peut être enseignée dans plusieurs salles
\item une salle peut recevoir plusieurs matières
\end{itemize}

Chaque salle va avoir un type spécifique correspondant aux différents types de cours (cours classe entière, TP, cours de demi classe).

\newpage
\section{Simplification du problème}
Afin de simplifier le problème, nous avons décidé de le résoudre sans tenir compte des contraintes imposées par les classes de M2. 
En effet, contrairement aux autres promotions, la M2 est composée de 4 classes ayant des programmes différents. Chacune de ces classes étant elle-même séparée en plusieurs spécialités ayant un tronc commun et un programme spécialisé.
Par conséquent nous avons décidé de commencer la résolution du problème en se focalisant sur les promotions ayant un programme unique.\\

De plus, nous avons décidé de nous affranchir des contraintes relatives aux salles, notamment en ce qui concerne la distinction entre les locaux de Montparnasse et d'Ivry-sur-Seine.\\

\newpage

\documentclass[12pt,a4paper,french]{article}

\usepackage[T1]{fontenc}
\usepackage[latin9]{inputenc}
\usepackage{fullpage}
\usepackage[french]{babel} 
\usepackage[official]{eurosym}
\usepackage{txfonts}
\usepackage{graphicx}
\usepackage{lastpage}
\usepackage{fancyhdr}
\usepackage{titlesec}
\usepackage{color}
\usepackage{setspace}
\usepackage[nottoc, notlof, notlot]{tocbibind}
\usepackage{hyperref}
\usepackage[french,ruled,vlined]{algorithm2e}
\usepackage{algorithmic}

\begin{document}

\title{Algorithmes -- Emploi du temps}
\author{Coudray -- Julien -- Tran}
\date{\today}
\maketitle

\section{Pr�-traitements}
Avant de commencer � r�aliser l'emploi du temps proprement parler, il faut v�rifier que nos donn�es d'entr�es ne vont pas nous mener directement vers une non solution.

\subsection{Le nombre de professeur}
La premi�re v�rification va concerner le nombre de professeur que nous avons en entr�e. La question qu'il faut se poser est : il y a t'il assez professeur pour donner chaque cours � chaque promotion?
Le principe de cette algorithme est de faire la somme des cr�neaux de disponibilit�s de l'ensemble des profs pouvant donner un cours et le comparer avec le nombre de classes qui va devoir suivre ce cours.\\

\begin{algorithm}
\caption{Pr�-traitement nombre de professeurs}
\begin{algorithmic}
\FORALL{$Courses$}
\STATE $idCourse \leftarrow id\_course[Courses]$
\STATE $idPromo \leftarrow id\_promo[Courses]$
\STATE $nbClasses \leftarrow$ nombre de classe de la promo $idPromo$
\FORALL{$Profs$}
\IF {$Profs$ donne le cours $idCourse$}
\FORALL {$ProfSlots$}
\IF {$Profs$ est displonnible}
\STATE $nbSlots \leftarrow nbSlots + 1$
\ENDIF
\ENDFOR
\ENDIF
\ENDFOR
\IF {$Courses$ est sur 4h}
\STATE $nbSlots \leftarrow nbSlots / 2$
\ENDIF
\IF{$nbClasses > nbSlots$}
\STATE display (Erreur sur le nombre de professeur pour la promo $idPromo$)
\STATE EXIT FAILURE
\ENDIF
\ENDFOR
\STATE display (Nombre de professeur ok)
\end{algorithmic}
\end{algorithm}

\newpage

\subsection{Le nombre de cours total sur le semestre}
La seconde v�rification va porter sur le nombre d'heure total que va suivre une classe sur l'ensemble du semestre. Ce nombre d'heure ne doit pas exc�der le nombre d'heure maximum par rapport au nombre total de cr�neaux du semestre. Le principe est de faire la somme de l'ensemble des cours que va avoir la classe et la comparer au nombre d'heure maximum du semestre.

\begin{algorithm}
\caption{Pr�-traitement nombre d'heure sur le semestre}
\begin{algorithmic}
\FORALL{$Classes$}
\STATE $listCourses \leftarrow$ ensemble des cours que suit une classe
\FORALL {$courses$ in $listCourses$}
\STATE $nbHours \leftarrow nbHours +$ nombre d'heure du cours $courses$
\ENDFOR
\IF {$nbHours$ > (nombre de semaine du semestre * nombre de cr�neaux par semaine * nombre d'heures par cr�neau)}
\STATE display(Erreur, trop d'heures pour la classe $Classes$)
\STATE EXIT FAILURE
\ENDIF
\ENDFOR
\STATE display(Nombre d'heure de cours ok)
\end{algorithmic}
\end{algorithm}

Une fois ces pr�-traitements r�alis�, nous pouvons commencer � r�aliser l'emploi du temps de l'�cole.

\newpage
\section{R�alisation de l'emploi du temps}
La r�alisation de l'emploi du temps se d�roule en plusieurs �tapes. Il sera r�alis� pour chaque promotion ind�pendamment. Une promotion ayant plusieurs classes qui vont suivre le m�me programme sur l'ensemble du semestre, avec le m�me nombre d'heure et la m�me liste de professeur pouvant enseigner la liste des mati�res. 
Puis pour chaque promotion, r�aliser un emploi du temps sur le semestre. A savoir planifier l'ensemble des cours avec une semaine de d�part et une semaine de fin.
Enfin, placer les cours sur les cr�neaux en parcourant semaines par semaines sur l'ensemble des classes simultan�ment.

\begin{algorithm}
\caption{Principe g�n�ral de conception de l'emploi du temps}
\begin{algorithmic}
\FORALL{$IdPromo$}
\FORALL{$Classes$}
\IF{$idPromo$ de Classes $= IdPromo$}
\STATE $idCourses \leftarrow$ liste de tous les cours que doivent suivre la promotion $IdPromo$
\STATE BREAK
\ENDIF
\ENDFOR
\STATE repartitionCoursSemestre($idCourses, programmeSemestre$)
\STATE repartitionCoursPromotions($IdPromo, programmeSemestre$)
\ENDFOR
\end{algorithmic}
\end{algorithm}

\newpage
\subsection{R�partition du programme sur le semestre}
La premi�re �tape de l'algorithme de r�solution de planification de l'emploi du temps de l'�cole est � partir de l'ensemble du programme de la promo de placer les cours � une date de d�but et de fin sur le semestre. Dans un premier temps il va falloir donner un trie d'insertion des cours sur le programme du semestre. Il faut s�parer les cours de deux et quatre heures, puis les trier dans l'ordre d�croissant du nombre de semaines sur lequel va se d�rouler le cours. Ensuite la r�partition des cours sur le semestre en essayant toujours de placer un cours � la suite d'un fini. Dans le cas o� il est impossible, le cours d�butera au d�but du semestre.

\begin{algorithm}
\caption{Algorithme principale de la r�partition des cours sur le semestre}
\begin{algorithmic}
\REQUIRE liste $idCourse$, liste $programmeSemestre$
\FORALL {$idCourse$}
\IF {$idCourse$ est un cours sur 2h}
\STATE $idCourses2 \leftarrow$ pushback $idCourses$
\ELSE
\STATE $idCourses4 \leftarrow$ pushback $idCourses$
\ENDIF
\ENDFOR
\STATE Trie de $idCourses2$ par nombre de semaine de cours d�croissant
\STATE Trie de $idCourses4$ par nombre de semaine de cours d�croissant
\STATE $idCourses$ est vider
\STATE $idCourses \leftarrow$ pushback $idCourses4$
\STATE $idCourses \leftarrow$ pushback $idCourses2$
\STATE $programmeSemestre \leftarrow$ repartitionDesCours($idCourses$)
\end{algorithmic}
\end{algorithm}

\end{document}

\section{Discussions}

Le problème étant vraisemblablement Np-difficile, nous doutons de la possibilité d'établir un algorithme de résolution exacte dans un temps raisonnable, étant donnée la taille des instances que nous avons observées. Nous avons considéré qu'une approche heuristique (voire méta-heuristique), offrant une solution réalisable mais non nécessairement optimale, serait plus appropriée pour des raisons pratiques évidentes.\\

Le temps nous faisant défaut, nous n'avons pas pu implémenter les contraintes liées à la M2 et résoudre le problème des salles et des locaux.\\

Par ailleurs, les TPs posent problème parce que cela revient à avoir plusieurs cours sur un même créneau et d'avoir plusieurs fois le même cours dans la semaine.\\
Le système des TP est un peu particulier : chaque série peut avoir 1 à 2 TP par semaine, et quoi qu'il arrive, il y aura toujours au moins une série de la classe dans chaque TP. Pour résoudre le problème des TP, on peut donc envisager de créer 2 sessions TP1 et TP2 à placer dans l'emploi du temps, et de réaliser la rotation des TP ultérieurement à l'aide d'un programme tiers.\\

Concernant le problème des salles, à partir de l'emploi du temps que nous avons créé, nous pouvons allouer à chaque créneau une salle aléatoirement jusqu'à avoir une répartition possible.\\


\newpage
\section{Conclusion}

A l'issue de ce projet, nous avons pu aboutir à une résolution heuristique du problème de planification dans le cadre de l'école.
Même si cette solution n'est pas optimale, elle constitue tout de même une base solide pour établir les emplois du temps des 4 premières promotions de l'école.\\

A partir de la saisie des données sur les professeurs, les promotions et les matières, le programme est capable de générer les emplois  du temps de chacune des classes et d'afficher les cours non placés le cas échéant, en proposant de les rajouter manuellement. Le programme propose également des fonctions de maintenance pour déplacer des cours en cas d'imprévu au fil de l'année.\\

Dans l'état actuel des choses, les données d'entrée/sortie du programme sont stockées dans des fichiers textes, sans aucune mise en forme. Afin de rendre le programme compréhensible et ergonomique, une interface graphique est en cours de réalisation dans le cadre d'un projet de M1.\\

\newpage
\addcontentsline{toc}{section}{Table des figures}
\listoffigures

 \listofalgorithms

\newpage
\bibliographystyle{unsrt}
\bibliography{biblio}


\end{document}