\documentclass{beamer}

\usepackage[frenchb]{babel}
\usepackage[T1]{fontenc}
\usepackage[utf8]{inputenc}

\usetheme{Warsaw}

\title{Réalisation d'un emploi du temps}
\author{Corentin Coudray - Christophe Julien - Nöel Tran}
\institute{ESME SUDRIA}

\begin{document}

\begin{frame}
\titlepage
\end{frame}

\section{Mise en forme des données}
\begin{frame}
5 fichiers d'informations : 
\begin{itemize}
\item Les professeurs
\item Les cours
\item Les classes
\item Les emplois du temps générés
\item La liste de cours non placé
\end{itemize}
\end{frame}

\subsection{Les professeurs}
\begin{frame}
4 informations : 
\begin{itemize}
\item l'identifiant
\item le nom
\item ses disponibilités
\item la liste des cours enseigné
\end{itemize}
\end{frame}

\subsection{Les cours}
\begin{frame}
3 informations : 
\begin{itemize}
\item l'identifiant
\item le nom
\item le nombre d'heure sur le semestre
\end{itemize}
\end{frame}

\subsection{Les classes}
\begin{frame}
5 informations : 
\begin{itemize}
\item l'identifiant
\item le nom
\item la promotion à laquelle elle appartient
\item les cours à recevoir
\item le nombre d'élève
\end{itemize}
\end{frame}

\subsection{Les emplois du temps}
\begin {frame}
Chaque classe à 1 fichier d'emploi du temps.\\
Il contient pour chaque créneau : 
\begin {itemize}
\item l'identifiant du cours
\item l'identifiant du professeur
\end{itemize}
\end{frame}

\subsection{Les cours non placés}
\begin{frame}
Les cours qui n'ont pu être placé seront listés dans un fichier.
Chaque cours aura : 
\begin{itemize}
\item l'identifiant du cours
\item l'identifiant de la promotion
\item l'identifiant du professeur
\item le numéro de la semaine
\end{itemize}
\end{frame}

\section{Génération de l'emploi du temps}
\begin {frame}
Génération de l'emploi du temps à partir des fichiers d'entrés.

3 étapes de conception : 
\begin{itemize}
\item les pré-traitements
\item emploi du temps semestriel général
\item emploi du temps spécifique pour chaque classe
\end{itemize}
\end{frame}

\end{document}